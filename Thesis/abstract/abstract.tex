\chapter*{Abstract}
Recently, the widespread growth of social media has shifted the way people explore and share information of interest. Part of this evolution is the event landscape increasingly augmented by user-generated content leading to vast amount of event-centric data. In today's Web, numerous are the services that provide facilities to organize and publish events, and to share related thoughts and captured media. However, the information about the events, the social interactions and the representative media are all spread and locked into the sites providing limited event coverage and no interoperability of the description. To fully benefit from the event, users are constrained to monitor different channels, some of which suffer from the information overload problem. 

The goal of this thesis is to provide a unified environment that provides broad event coverage along with complete description and illustrative media, and to investigate efficient approaches that can benefit content personalization. The major challenge is to face the complex nature of events as multifaceted, ephemeral and social entities.

Various distributed platforms host a wide variety of scheduled events along with related media and background knowledge, making the user-contributed Web a primary source of information about any real world happening. Mining in real-time the connections between these heterogeneous and spread data fragments is a key factor to improve data quality and to enable opportunistic discovery of events. Towards this goal, we integrate different sources using Linked Data, so that we can explore the information with the flexibility afforded by Semantic Web technologies. More precisely, we leverage the wealth of information derived from event-based services, media platforms and social networks to build a Web environment that allows users discovering meaningful connections between events, media and people.

On the other hand, users tend to be overwhelmed by the massive amount of information available in event-based services. This fact requires valuable solutions that cope with the information overload. In particular, recommendation and community detection are two promising solutions that have been widely investigated in research. Yet their applications in event domain are still elusive. Thus as a first solution, we propose a hybrid recommender system that capitalizes on ontology-based event representation along with the collaborative filtering techniques. Second, we propose an approach to discover topical communities in event-based social network. Both the network links and the event topics are examined during the clustering process in which the quality function is the so-called semantic modularity.