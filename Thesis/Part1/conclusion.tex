\chapter*{Conclusion of Part~\ref{part:dataset_profiling}}

In this part, we surveyed the landscape of various models and vocabularies that described datasets on the web. Since establishing a common vocabulary or model is the key to communication, we identified the need for an harmonized dataset metadata model containing sufficient information so that consumers can easily understand and process datasets. We have identified four main sections that should be included in the model: resources, groups, tags and organizations. Furthermore, we have classified the information to be included into eight types. Our main contribution is a set of mappings between each properties of those models. This has lead to the design of HDL, an harmonized dataset model, that takes the best out of these models and extends them to ensure complete metadata coverage to enable data discovery, exploration and reuse.

At the moment, HDL is available as a hierarchical JSON file. As part of our future work, we plan to refine HDL and present it as a fully fledged OWL ontology. At the moment, HDL contains some values that were frequently defined in CKAN extras fields. However, we plan to broaden our analysis of these values by running Roomba on additional portals and present the top results as enumerations, ensuring a fine-grained representation of a dataset. We further plan to create mappings between HDL and all the various models to ensure full compatibility. These mappings, for example, can be used to extend Roomba allowing it to perform metadata profiling on other portals like DKAN. Finally, we plan to create a set of supporting tools that allow validation of generation of HDL profiles.



