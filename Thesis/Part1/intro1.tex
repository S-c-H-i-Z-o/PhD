\chapter*{Overview of Part \ref{pa:part1}}

In Part~\ref{pa:part1}, we focus on the development of a framework that retrieves and links event-centric information derived from event directories, media platforms and social networks. We capitalize on Semantic Web technologies to ensure a flexible and large-scale integration of disparate data sources, some of which overlap in their coverage. The goal is to provide a support for exploring and selecting events associated with media, and for discovering meaningful connections between them. 
\\
\\
In Chapter~\ref{ch:data-aggregation}, we present the different steps in building a large dataset called EventMedia which is composed of event descriptions associated with media. These steps include data aggregation and structuring into a unified knowledge model using ontologies. One fundamental requirement is to set a flexible architecture, so that it can easily support the addition of event and media Web services.
\\
\\
In Chapter~\ref{ch:data-reconciliation}, we focus on the fourth element of the Linked Data principles which is to link data together. The goal is to explore the implicit overlap of the disparate data sources trying to overcome some well-known types of data heterogeneity. We mainly investigate the following questions: what heuristics are suitable to reconcile event-centric information in Linked Data? How to align structured events with unstructured media content? 